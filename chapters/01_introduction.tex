% !TeX root = ../main.tex
% Add the above to each chapter to make compiling the PDF easier in some editors.

\chapter{Introduction}\label{chapter:introduction}

\begin{enumerate}
	\item Medical Imaging
	\item Global Warming
	\item Optional: Emission Inventories
\end{enumerate}

\section{Motivation}
Bottom-up emission estimations are inaccurate.
Instead, top-down approaches could be used.
... et al. demonstrate that top-down approaches can be represented as an inverse problem with measurements $y$ and the emission field as $x$.
The transport of ghg molecules can be modeled as a linear map $A$ between $y$ and $x$, resulting in the following inverse problem:
\begin{equation}
	y = Ax
\end{equation}
In their paper, ... et al. show that using

Three reasons for bottom-up approach:
\begin{itemize}
	\item find unknown emitters
	\item determine diff. between bottom-up \& top-down
	\item find emitters not captured by inventories
\end{itemize}

\section{Research Questions}
This thesis aims at answering some questions:

\begin{itemize}
	\item How well can a variaional autoencoder capture a low dimensional representation of area sources in emission inventories?
	\item What is the effect of the dimension of that low dimensional space?
	\item Is the variational autoencoder generazible to all urban emission fields or should it be specialized on one city?
	\item How well does such a variational autoencoder perform in atmospheric inversion of area sources as a downstream task compared to state of the art approaches?
\end{itemize}
