% !TeX root = ../main.tex
% Add the above to each chapter to make compiling the PDF easier in some editors.

\chapter{Introduction}\label{chapter:introduction}

\section{Motivation}
Bottom-up emission estimations are inaccurate.
Instead, top-down approaches could be used.
... et al. demonstrate that top-down approaches can be represented as an inverse problem with measurements $y$ and the emission field as $x$.
The transport of ghg molecules can be modeled as a linear map $A$ between $y$ and $x$, resulting in the following inverse problem:
\begin{equation}
	y = Ax
\end{equation}
In their paper, ... et al. show that using

Three reasons for bottom-up approach:
\begin{itemize}
	\item find unknown emitters
	\item determine diff. between bottom-up \& top-down
	\item find emitters not captured by inventories
\end{itemize}

\subsection{Questions}
This thesis aims at answering some questions:
\begin{itemize}
	\item Can variational autoencoders learn a low dimensional representation of emission inventories that allows using them in the context of compressed sensing for emissions?
	\item What is the dimension of that low dimensional representation?
	\item How does the model compare to previous approaches? How does it perform for less or more number of measurements?
	\item How does it perform for very high resolution emission fields, like 100m resolution?
\end{itemize}

Technical University Munich \gls{TUM} \Gls{TUM}
