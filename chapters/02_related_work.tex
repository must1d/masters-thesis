% !TeX root = ../main.tex
% Add the above to each chapter to make compiling the PDF easier in some editors.

\chapter{Related Work}\label{chapter:related_work}

\section{Emission Inventories}
Questions:
\begin{enumerate}
    \item what are Emission Inventories
    \item what are the problems with bottom-up approaches for GHG emissions? 
    \subitem may miss
    \item what are typical approaches to top-down?
    \item what are the challenges in top-down approaches?
\end{enumerate}
Then go over to Benjis work who has shown that 
For this work, same assumptions as in Benjis work hold, i.e. background GHG emissions are ignored.

\section{Compressed Sensing}
There are three main approaches to alleviate sparsity constraint.
1) transforms, such as Wavelet
2) only searching for unknown emissions and take esimtations as basis
3) Deep learning based approaches

\section{Deep Learning for Inverse Problems}
The field of medical imaging has made advancements in the inverse problems using deep learning.
A review of different deep learning approaches for CS is given in \parencite{ReviewCSUsingAI}
They mention Bora et al \parencite{CSUsingAI}.
Their approach has the limitation that the reconstruction is constrained to the range of the generator.
This, however, can improved by also taking into account other things: \parencite{SparseCSUsingAI}.

\section{Generative Models}

\section{Contributions}
