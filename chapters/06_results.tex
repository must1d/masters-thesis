% !TeX root = ../main.tex
% Add the above to each chapter to make compiling the PDF easier in some editors.

\chapter{Results}\label{chapter:results}
In this chapter, the following experiments are performed.
For all experiments the test data is used.
Then, an inverse problem is generated.
This is run x times.

\section{Latent Dimension}
The latent dimension of the VAE has an effect of the representational error.
With an increase of the latent dimension, the VAE is able to represent more information.
This means the range of the generator is bigger.
The representational error, i.e. the error observed if the complete field is sensed, is smaller.
However, with a bigger range, the search for a point becomes more difficult.
This results in the following [to fill].
A smaller latent dimension will have a smaller error for few measurements
A bigger latent dimension will have a smaller error for many measurements.

\section{Fine-tuned Models vs Base Models}
\section{Generative Models vs Lasso}
Make a graph that keeps num measurements fixed and only varies the noise.

\section{Reconstruction of Unknown Emitters}

IMPORTANT:
Mention the effect of noise

\section{Gaussian Plume Model}
To fill.

Make a plot that has x axis snr and y axis relative error just like in Benjis plot.
Do this for the 3 cities.
