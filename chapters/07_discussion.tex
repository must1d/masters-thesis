% !TeX root = ../main.tex
% Add the above to each chapter to make compiling the PDF easier in some editors.

\chapter{Discussion}\label{chapter:discussion}

\section{Model}
The model is far from ideal.
By using MCMC, it was obvious that the latent variable could not be modeled as Gaussian.
By investigating the latent variable z it is obvious that the latent space does not follow a perfect Gaussian distribution.
We suspect that this has to do with the lack of distinct training samples.
Time sclaing does not fundamentally change the training sample.
Thus, fundamentally only n cities are contained in the training data.
We suspect that by gathering more training data, the model is able to capture the shape of the city emissions better.
Then, one could try to use algorithms such as MCMC.

This is also supported by the fact that a regularization factor of 0 works the best.

\section{Future Applciation}
This thesis presents the idea of using machine learning based solving of inverse problems with a technique from 2017.
The field of ML based inverse problems has made significant progress in the last year with research progress in generative AI.
This thesis only shows that such techniques are applicable to the problem of reconstructing emitters from measurements.
Thus, thie thesis opens the gates to exploring more of these approaches and apply state of the art techniques.
