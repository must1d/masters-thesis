% !TeX root = ../main.tex
% Add the above to each chapter to make compiling the PDF easier in some editors.

% \chapter{Code for Generation of Gaussian Plume Footprints}\label{chapter:gaussian_plume_footprints}

% \begin{lstlisting}[language=Python, caption={Example of Python code}]
% class WindField:
%     def __init__(
%         self,
%         speed: float = 0.3,
%         omega: float = 0.1,
%         static_time: float = 10.0,
%     ) -> None:
%         self._speed = speed
%         self._omega = omega
%         self._static_time = static_time

%     def __call__(self, x: np.ndarray, y: np.ndarray, t: float) -> tuple[np.ndarray, np.ndarray]:
%         omega = self._omega if t > self._static_time else 0

%         t_ = t - self._static_time

%         u = 2 * self._speed * np.sin(0.1 * np.pi * x + np.pi * omega * t_) * np.cos(omega / 2 * t_)
%         v = self._speed * np.cos(0.1 * np.pi * y + np.pi * omega * t_) * np.cos(omega / 2 * t_)
%         return u, v
% \end{lstlisting}

% \begin{lstlisting}[language=Python, caption={Example of Python code}]
% class AdvectionDiffusionEquation:
%     D = 0.001

%     def __init__(
%         self,
%         source_x: int,
%         source_y: int,
%         simulation_width: int,
%         simulation_height: int,
%         wind_field: WindField,
%     ) -> None:
%         self._nx = simulation_width
%         self._ny = simulation_height

%         self._dx = 1 / self._nx
%         self._dy = 1 / self._ny

%         x = np.linspace(0, 1, simulation_width)
%         y = np.linspace(0, 1, simulation_height)
%         self._X, self._Y = np.meshgrid(x, y)

%         self._source = self._get_source(source_x, source_y)

%         self._wind_field = wind_field

%     def _get_source(self, x: int, y: int) -> np.ndarray:
%         s = np.zeros((self._ny, self._nx))
%         s[y, x] = 1
%         return gaussian_filter(s, sigma=1)

%     def __call__(self, t: float, u_flat: np.ndarray) -> np.ndarray:
%         u = u_flat.reshape((self._nx, self._ny))

%         # Compute second-order derivatives (Diffusion terms)
%         du_dx2 = (np.roll(u, -1, axis=0) - 2 * u + np.roll(u, 1, axis=0)) / self._dx**2
%         du_dy2 = (np.roll(u, -1, axis=1) - 2 * u + np.roll(u, 1, axis=1)) / self._dy**2

%         # Compute first-order derivatives (Advection terms)
%         du_dx = (np.roll(u, -1, axis=0) - np.roll(u, 1, axis=0)) / (2 * self._dx)
%         du_dy = (np.roll(u, -1, axis=1) - np.roll(u, 1, axis=1)) / (2 * self._dy)

%         v_x, v_y = self._wind_field(self._X, self._Y, t)

%         # Note inverted sign for wind speed to simulate backwards in time computation
%         du_dt = v_x * du_dx + v_y * du_dy + self.D * (du_dx2 + du_dy2) + self._source

%         # Apply zero-gradient boundary conditions
%         du_dt[0, :] = du_dt[-1, :] = 0
%         du_dt[:, 0] = du_dt[:, -1] = 0

%         return du_dt.flatten()
% \end{lstlisting}

% \begin{lstlisting}[language=Python, caption={Example of Python code}]
% class GaussianPlumeModel:
%     WIDTH = HEIGHT = 32

%     UP_SAMPLING_FACTOR = 2

%     SIMULATION_WIDTH = UP_SAMPLING_FACTOR * WIDTH
%     SIMULATION_HEIGHT = UP_SAMPLING_FACTOR * HEIGHT

%     STATIC_TIME = 5

%     TIME_PER_MEASUREMENT = 0.2

%     dt = 0.1

%     def get_sensitivities_for_sensor(self, sensor_x: int, sensor_y: int, num_measurements: int) -> list[np.ndarray]:
%         u_0_flat = np.zeros((self.SIMULATION_HEIGHT, self.SIMULATION_WIDTH)).flatten()

%         pde = self._setup_advection_diffusion_equation(sensor_x=sensor_x, sensor_y=sensor_y)

%         t_max = self.STATIC_TIME + self.TIME_PER_MEASUREMENT * num_measurements

%         sol = solve_ivp(
%             pde,
%             [0, t_max],
%             u_0_flat,
%             method="RK45",
%             t_eval=np.arange(0, t_max, self.dt),
%         )

%         u_sol = sol.y.reshape((self.SIMULATION_HEIGHT, self.SIMULATION_WIDTH, -1))

%         down_sampled_array = self._down_sample_solution(u_sol)

%         relevant_time_stamps = [
%             int((self.STATIC_TIME + i * self.TIME_PER_MEASUREMENT) // self.dt) for i in range(num_measurements)
%         ]

%         return [down_sampled_array[:, :, i] for i in relevant_time_stamps]

%     def _setup_advection_diffusion_equation(self, sensor_x: int, sensor_y: int) -> AdvectionDiffusionEquation:
%         wind_field = WindField(static_time=self.STATIC_TIME)
%         return AdvectionDiffusionEquation(
%             source_x=self.UP_SAMPLING_FACTOR * sensor_x + self.UP_SAMPLING_FACTOR // 2,
%             source_y=self.UP_SAMPLING_FACTOR * sensor_y + self.UP_SAMPLING_FACTOR // 2,
%             simulation_height=self.SIMULATION_HEIGHT,
%             simulation_width=self.SIMULATION_WIDTH,
%             wind_field=wind_field,
%         )

%     def _down_sample_solution(self, sol: np.ndarray) -> np.ndarray:
%         return resize(sol, (self.HEIGHT, self.WIDTH), anti_aliasing=True)
% \end{lstlisting}

\chapter{Code Availability}
The code developed for this research, as well as the model weights, are publicly available on GitHub at \url{https://github.com/tum-esm/inventory-embeddings}